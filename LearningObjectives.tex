\documentclass[12pt,landscape]{article}
\usepackage[margin=1in]{geometry}
\usepackage{graphicx}
\usepackage{multirow}
\usepackage{multicol}
\usepackage{setspace}

\setlength\parindent{0pt}
\usepackage{hyperref}
\usepackage{amssymb, fancyhdr, comment}
\pagestyle{fancy}

\setlength{\columnsep}{2cm}
%\setlength{\columnseprule}{0pt}

\begin{document}
\chead{Math 121 - Learning Objectives}
\begin{multicols}{2}
\begin{enumerate}
% NEW
\item Demonstrate an understanding of the underlying concept of a limit using graphs or tables.\\
%
\item Evaluate a limit using the limit laws.\\
\item Evaluate a limit at infinity.\\
\item Evaluate an infinite limit.\\
% DELETE
% \item Use limits to determine the horizontal and vertical asymptotes of a function.\\
%
\item Show, via the definition, that a function is continuous at a point.\\
\item Apply the Intermediate Value Theorem to show a function has a zero on a given interval.\\
% NEW
\item Demonstrate an understanding of the underlying concept of a derivative using graphs or tables.\\
%
\item Calculate the derivative of a polynomial function directly from the definition.\\
\item Calculate the derivative of a polynomial function using the power rule.\\
\item Calculate the derivative of a trigonometric function.\\
\item Calculate the derivative of a function using the product rule.\\ 
\item Calculate the derivative of a function using the quotient rule.\\
\item Calculate the derivative of a function using the chain rule.\\
\item Calculate the derivative of a function using a combination of the power, product, quotient, and/or chain rule.\\
\item Find the equation of a tangent line to a curve at a given point.\\
% NEW
\item Use a linear approximation to approximate a nonlinear function\\
%
\item Correctly find the derivative of an implicit function.\\
\item Correctly set up a problem involving at least two related rates.\\
\item Solve a problem involving at least two related rates.\\
\item Identify the intervals on which a function is increasing and/or decreasing.\\
\item Identify the intervals on which a function is concave up and/or concave down.\\
\item Find all critical values of a function.\\
% DELETE
%\item Use the 1st derivative test to classify extrema of a function. \\
%\item Use the 2nd derivative test to classify extrema of a function.\\
% NEW
\item (NEW) Use the 1st and/or the 2nd derivative test to classify extrema of a function. \\
%
\item Apply the Extreme Value Theorem to a problem.\\
\item Apply the Mean Value Theorem to a problem.\\
\item Correctly set up an optimization problem using the methods of Calculus.\\
\item Solve an optimization problem using the methods of Calculus.\\
% MODIFIED
\item Calculate an antiderivative (indefinite integral) of a polynomial function.\\
\item Calculate an antiderivative (indefinite integral) of a trigonometric function.\\
%
\item Use a finite summation to approximate the area under a curve.\\
% DELETE
%\item Calculate a definite integral using the Riemann Sum.\\
\item Evaluate a definite integral using the Fundamental Theorem of Calculus.\\
% DELETE (but incorporated into the antiderivative items)
% \item Evaluate an indefinite integral.\\
\item Evaluate an indefinite integral using substitution.\\
\item Calculate the derivative of a logarithmic function.\\
\item Calculate the derivative of an exponential function.\\
% DELETE
%\item Calculate a derivative using logarithmic differentiation.\\
\item Calculate the derivative of an inverse trigonometric function.\\
\item Evaluate a limit using L'Hospital's rule.\\
% NEW
\item Calculate an integral resulting in a logarithmic function.\\
\item Calculate an integral involving an exponential function.\\
%
\end{enumerate}
\newpage 

\end{multicols}
\end{document}
